\chapter{Bevezetés}
\label{ch:intro}

Az író eszközei az utóbbi évtizedekben -- habár más formában -- ugyanúgy az írott szöveg egy-egy formájában testesültek meg. A történetek számtalan karakterének egyéni céljait, érzéseit, állapotát követni fejben lehetetlen. A karakterlapok pedig néha nem árulnak el minden információt, összedőlnek saját súlyuk alatt.

Ez az alkalmazás nem hivatott lecserélni a meglévő módszereket, hanem egy új eszközként lépne az író repertoárjába. Olyan, egyébként írásban nehezen követhető, információt szolgáltat vizuálisan mellyel további segítséget tud nyújtani a történet konzisztenciájának megőrzésére, az ellentmondások elkerülésére. Ilyen például a karakterek térbeli pozíciójának és sebességének a nyilvántartása, mellyel pontosan meghatározható mennyi időnek kell elkelnie, hogy két, távol lévő karakter találkozhasson. Hordozott információik bármikor megváltozhatnak, és ezek befolyásolhatják a jövőbeli interakcióikat. Ezeket pedig habár egy papírlapra is fel lehetne vinni, az kevés teret ad a változtatásra. Egy, a történet elején lévő pillanat megváltoztatása, kihathat mindenre ami utána jön. Valamint vizuális visszajelzést is nyújt azzal, hogy lejátszhatóvá teszi a felvitt adatokat. Webalkalmazás lévén pedig könnyen hozzáférhető bárki számára.

Az alkalmazás fő szolgáltatása a szereplők sebességének mindenkori limitálása és az információ-akkumulálás. Karaktereinkkel bármikor történhet valami, hozzájuthatnak új információkhoz, esetleg el is felejthetik azt. Ez pedig kihathat bármire ami később történik. Ezért a felhasználó akármelyik időponton ha megtekint egy szereplőt, az addigi összes változtatás eredményét fogja látni.
