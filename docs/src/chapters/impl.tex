\chapter{Fejlesztői dokumentáció}
\label{ch:impl}

Az alkalmazás a könnyű megközelíthetőség miatt web alapúan kerül megvalósításra. Ennek számos előnye van. Az alkalmazás nem igényel telepítést, bármilyen számítógépről, bármikor elérhető. A már ismert webes technológiákat alkalmazhatjuk. És Progresszív Web Alkalmazás (PWA \cite{PWA}) lévén az első megnyitást leszámítva aktív internetelérés sem kell a használatához.

\section{Megoldási terv}

Az alkalmazás fő eleme a projekt amit Lore-nak nevezünk, ennek an egy aznosítója mivel dokumentum szintű objektumról van szó, egy neve és ebben tároljuk a világ adatait melyek annak mérete és neve. Valamint ehhez a dokumentumhoz tartozik a világ textúrája is.

Egy projecthez tartoznak még szereplők amik szintén dokumentum szintű objektumok, saját kollekcióval. Ebben tároljuk a szereplő állapotait egy AVL fában \cite{AVL} amiben a kulcs az idő, ezzel biztosítva, hogy minden esemény időben sorban legyen, és gyorsan elérhetőek legyenek.

Egy állapotdelta csak az előző állapotdeltákhoz viszonyított különbséget tartja nyilván, kivéve a pozíciót, ami minden esetben kötelező. Ezeket a deltákat a szereplők és a jelenlegi idő csöveinek összevezetésével egy aggregátoreljárás értékeli ki. Erre feliratkozva bármi azonnal hozzáférhet a legfrissebb adatokhoz ami csak egy helyen, egyszer van emiatt kiszámolva.

\subsection{Az adatmodell}
\begin{plantuml}
@startuml
skinparam monochrome true

note left of Lore: Tartozhat minden példányhoz egy\nRxDocument 'texture' azonosítóval

interface Lore {
	id: string
	name: string
	planet: Planet
}

interface Planet {
	name: string
	radius: number
}

interface Actor {
	id: string
	loreId: string
	states: Tree
}

note left of Tree: Nem perzisztálható de szérializálható
class Tree {
	root: Node
}

class Node {
	key: number
	value: ActorDelta
}

note left of ActorDelta: Egy AVL\nfában helyezkednek el
interface ActorDelta {
	position: Vec3
	name?: string
	maxSpeed?: number
	color?: string
	properties?: Array<Property>
}

interface Vec3 {
	x: number
	y: number
	z: number
}

interface Property {
	key: string
	value: string
}

Lore -- Planet
Lore --o Actor
Actor -- Tree
Tree --* Node
Node -- ActorDelta

ActorDelta --* Property
ActorDelta -- Vec3

@enduml
\caption{Az adatmodell}
\label{fig:data-model}
\end{plantuml}
%	\caption{Az adatmodell}
%	\label{fig:data-model}
%\end{figure}

\subsection{A reaktív modell}

A klasszikus imperatív modell helyett az alkalmazás jelentős részében reaktív komponenseket használunk amik alapjául az RxJS \cite{RxJS} csomag szolgál. Ez azt jelenti, hogy klasszikus adattagok helyett, adatforrásokat, megfigyelhető objektumokat (Lásd Megfigyelő programtervezési minta \cite{ObserverPattern}), függvényhívások helyett pedig mellékhatásokat használunk.

Ez a minta lehetőséget ad arra, hogy bármilyen nézet a böngészőben ami felhasznál valamilyen adatot, azonnal frissüljön mikor a forrása megváltozik.

\begin{figure}[h!]
	\centering
	\begin{plantuml}
		@startuml
		skinparam monochrome true
		hide empty description
		[*] -> Observable : Új adat kerül a megfigyelhető objektumba
		Observable --> Pipes
		Pipes : A csövek formában és időben is \nmegváltoztathatják a bennük átfolyó adatot
		Pipes --> SideEffects : A transzformáció közben mellékhatásokat is definiálhatunk
		Pipes --> ObserversA
		Pipes --> ObserversB
		Pipes --> ObserversC : Majd pedig megérkeznek minden feliratkozóhoz

		SideEffects  --> [*]
		ObserversA --> [*]
		ObserversB --> [*]
		ObserversC --> [*] : És kifejtik hatásukat
		@enduml
	\end{plantuml}
	\caption{Adatfolyam a megfigyelő modellben}
	\label{fig:observer-pattern}
\end{figure}

Kétfajta adatot különböztetünk meg, perzisztens és temporális. Perzisztens adatnak minősül minden ami egy projekthez tartozik. Temporális meg az amiket az előbbi manipulálásakor használunk fel. Például projekthez tartozó adat az, hogy egy szereplőnk egy adott időpillanatban hol tartózkodik, de az nem, hogy a kurzor épp ezen az időpillanaton áll.

A perzisztens a böngészők beépített noSQL adatbázisában, az IndexedDB-ben kerülnek perzisztálásra, dokumentumok formájában. Az adatbázis elérésehez, és ezzel az alkalmazás reaktív modelljének alapkövének RxDB-t \cite{RxDB} használunk.

Ez a csomag lehetővé teszi, hogy adatok beszúrásakor/módosításakor/törlésekor ezek a folyamatok igéretekként (Lásd Promise \cite{Promise}) jelenjenek meg melyek teljesüléséről értesítést kapunk. Ezzel lehetővé téve olyan featureöket mint, hogy jelezni a felhasználónak, hogy egy adott elem adatainak forrása éppen mentés alatt áll, és a szerkesztését letiltani. A legfőbb előnye viszont az, hogy bármilyen adatbázis kollekciót, lekérdezést megfigyelhetünk. És amint változás történik az adatbázisban, ezek a megfigyelhető Query-k újra jeleznek, frissítve minden mást ami rájuk épül.

A temporális adatokat viszont felesleges adatbázisban tárolni csak azért, hogy reaktív módon kezelhessük őket. Lehetőség van minden ilyen adatnak egy saját forrást (Subject) biztosítani. De mi van ha két adat függ egymástól? Csöveken keresztül futtassuk őket össze, válasszuk szét majd az így keletkező csöveket használjuk ezentúl? Mi van ha erre a változtatásra később kerül sor? Mi van ha valamit elfelejtünk átállítani, hogy a direkt-forrás helyett a transzformált csövet használják? Ezek a problémák és még sok más megoldására jön képbe az NgRX \cite{NgRx} csomag és a Redux (\cite{Redux}) architektúra amit implementál.

\begin{figure}[h!]
	\centering
	\begin{plantuml}
		@startuml
		skinparam monochrome true
		hide empty description
		[*] -d-> View : Felhasználói interakció
		View -d-> Dispatcher : Akció
		Dispatcher -l-> Effects : Akció
		Effects -r-> Dispatcher : Akció
		Effects -l-> [*] : Mellékhatás

		Dispatcher --> Reducer : Akció
		CurrentState -> Reducer : Az előző állapot
		Reducer --> Store : Új állapot készül
		Store -u-> View : Értesíti

		@enduml
	\end{plantuml}
	\caption{A Redux architektúra}
	\label{fig:redux-architecture}
\end{figure}

A lényeg a globális, nem módosítható (Csak cserélhető) állapot, az úgynevezett "igazság egyetlen forrása" ("Single source of truth") és az egyirányú adatfolyam (Unidirectional data-flow). Ennek hála egy átlátható, eseményeiben könnyen megjósolható programot kapunk, cserébe viszont sok kódot igényel.
Ezt egy központi Store service-ből tudjuk elérni, a könnyebb interakció miatt pedig egy Facade\cite{Facade} Service-t húzunk elé.

\section{Alkalmazás Architektúra}

Egy Angular alkalmazás modulokra van bontva. Minden modul saját Injection Scope-al rendelkezik, a felhasznált Service-ek pedig egyediek Injection Scope-onként. Mivel ez az alkalmazás csupán egy funkciót lát el megelégedhetnénk egy modullal is, de jobb felhasználói élményt érhetünk el ha az első töltést felgyorsítjuk azzal, hogy az applikációnk első töltési pontját kiürítjük, a tényleges applikációt pedig a gyökér routing pont mögé tesszük. Ez lusta módon \cite{LazyLoad} fogja betölteni a tényleges tartalommal rendelkező modult, nem blokkolva az alap modul tartalmának festését a DOM-ba. Az alap modul így taralmazhat egy egyszerű töltési animációt amit csak addig jelenítünk meg amíg a router tölti az almodult.

\begin{figure}[h!]
	\centering
	\begin{plantuml}
		@startuml
		skinparam monochrome true

		[-> AppModule: Applikáció indításakor
		activate AppModule
		AppModule -> AppModule : Első festés, LoreModule Lazy töltése

		activate  AppModule
		AppModule -> LoreModule
		deactivate AppModule

		activate LoreModule

		LoreModule ->]
		LoreModule <-]


		AppModule <- LoreModule: Kilépés
		deactivate LoreModule

		[<- AppModule: Kilépés
		deactivate AppModule

		@enduml


	\end{plantuml}
	\caption{Lusta betöltés}
	\label{fig:lazy-loading}
\end{figure}

\subsubsection{Servicek}

A serviceink első injectionnél inicializálódnak, Singleton \cite{Singleton} módon viselkednek. Ez tökéletes az adatbázis gyors bootstrappeléséhez.

\paragraph{DatabaseService}

Komponenseink adatait Service-ekből fogjuk kinyerni. Két adatforrást definiáltunk,







































\subsection{Technológiák}

\begin{table}[H]
	\centering
	\begin{tabular}{ | m{0.20\textwidth} | m{0.80\textwidth} | }
		\hline
		\textbf{Csomag} & \textbf{Szerep} \\
		\hline \hline
		\emph{Angular} \cite{Angular} & A fő keretrendszer, biztosítja a build eszközöket mint Webpack. \\
		\hline
		\emph{RxDB} \cite{RxDB} & Reaktív interfészt biztosít a böngészők IndexedDB adatbázisaihoz. \\
		\hline
		\emph{NgRx} \cite{NgRx} & Reaktív állapot menedzsment. \\
		\hline
		\emph{Three.js} \cite{Three} & WebGL könyvtár a 3D grafikai elemekhez. \\
		\hline
		\emph{Tween.js} \cite{Tween} & Interpolációs könyvtár. \\
		\hline
		\emph{FontAwesome} \cite{FontAwesome} & Ikon könyvtár.  \\
		\hline
		\emph{TypeScript} \cite{TypeScript} & Típusos JavaScript kiegészítő.  \\
		\hline

		\emph{Sass} \cite{Sass} & CSS kiegészítő könyvtár.  \\
		\hline
	\end{tabular}
	\caption{Az alkalmazás technológiái}
	\label{tab:technologies}
\end{table}

\begin{table}[H]
	\centering
	\begin{tabular}{ | m{0.20\textwidth} | m{0.80\textwidth} | }
		\hline
		\textbf{Eszköz} & \textbf{Szerep} \\
		\hline \hline
		\emph{NPM} \cite{NPM} & JavaScript csomagkezelő eszköz.  \\
		\hline
		\emph{GitHub} \cite{Github} & Online git repository.  \\
		\hline
		\emph{Travis-CI} \cite{Travis} & Online build és deployment.  \\
		\hline
	\end{tabular}
	\caption{Devops eszközök}
	\label{tab:technologies}
\end{table}


-






A létrehozott projecteket egy Lore nevű dokumentum fogja tárolni




\section{Adat Modell}










Az alkalmazás induláskor az alábbi folyamat indul el, ez mindig biztosítani fog egy élő adatbáziselérést az applikáció többi részére.

\begin{plantuml}
	@startuml
	skinparam monochrome true
	[*] --> ApplicationBootstrap

	state ApplicationBootstrap {
		ApplicationBootstrap : Angular Inner Sequences
	}
	ApplicationBootstrap --> DatabaseBootstrap

	state DatabaseBootstrap {
		[*] --> InitializeCollections
		InitializeCollections : Lore
		InitializeCollections : Actor
		InitializeCollections : ActorDelta
		InitializeCollections --> InitializeExample
		InitializeExample : Az alap project létrehozása
		InitializeExample --> [*]
		note right of InitializeCollections : Ez a folyamat csak akkor\nhozza létre a kollekciókat\nha szükség van rá
	}

	DatabaseBootstrap --> DatabaseBootstrap : On Fail

	DatabaseBootstrap --> [*]

	@enduml
\end{plantuml}



\begin{plantuml}
	@startuml
	skinparam monochrome true
	DatabaseBootstrap --> LoresLoaded: LoadLores

	state LoresLoaded {
	}

	LoresLoaded --> LoreSelected: ChangeSelectedLore

	ChangeSelectedLore --> asd
	LoreSelected --> [*]


	@enduml
\end{plantuml}


\begin{plantuml}
	@startuml
	skinparam monochrome true
	[-> Database : Database Initialized
	== Initialization ==
	group LoadLores
		Database --> LoreState: Single find
	end
	== Repetition ==

	group State refresh hooks
		Database -> LoreState: On Insert
		Database -> LoreState: On Update
		Database -> LoreState: On Delete
	end

	group Database manipulation
		DatabasStateFacade -> Database: Do Insert
		DatabasStateFacade -> Database: Do Update
		DatabasStateFacade -> Database: Do Delete
	end

	LoreState --> LoreState: ChangeSelectedLore

	@enduml
\end{plantuml}

Az al-állapotokból kifele induló akciók mindíg a globális állapotteret jelentik, amik vissza fognak érkezni egy másik al-állapotba, vagy akár ugyanebbe.



\begin{figure}[h!]
	\centering
	\begin{plantuml}
		@startuml
		skinparam monochrome true
		[-> Database : Database Initialized
		== Initialization ==

		Database --> LoreState: Load Lores
		LoreState -> LoreState: Shim each entry
		LoreState ->]: loadLoresSuccess
		@enduml
	\end{plantuml}
	\caption{A globális állapottér bootstrap folyamata}
	\label{fig:global-state-bootstrap}
\end{figure}

\begin{figure}[h!]
	\centering
	\begin{plantuml}
		@startuml
		skinparam monochrome true
		note left of LoreState : updateInitialSelectedLore
		[-> LoreState : loadLoresSuccess
		LoreState -> LoreState: Select the first one
		LoreState ->] : changeSelectedLore
		@enduml
	\end{plantuml}
	\caption{Az első kiválaszott project kiválasztása}
	\label{fig:global-state-bootstrap}
\end{figure}

\begin{figure}[h!]
	\centering
	\begin{plantuml}
		@startuml
		skinparam monochrome true
		Database --> LoreState: createLoreSuccess
		LoreState ->]: changeSelectedLore
		@enduml
	\end{plantuml}
	\caption{Új projekt létrehozásakor automatikusan kiválasztásra kerül}
	\label{fig:global-state-bootstrap}
\end{figure}


\begin{figure}[h!]
	\centering
	\begin{plantuml}
		@startuml
		skinparam monochrome true
		[-> LoreState: changeSelectedLore
		LoreState ->]: changeSelectedLoreSuccess
		[-> ActorState: changeSelectedLoreSuccess
		Database --> ActorState: connection
		ActorState -> ActorState: loading everything into state
		ActorState ->]: loadActorsSuccess
		[-> ActorState: loadActorsSuccess
		loop for each actor object that has been loaded
		ActorState ->]: loadActorDeltasForActor
		end
		@enduml
	\end{plantuml}
	\caption{Projekt kiválasztásakor betöltődik az összes szereplő és azok összes deltái}
	\label{fig:global-state-bootstrap}
\end{figure}

\begin{figure}[h!]
	\centering
	\begin{plantuml}
		@startuml
		skinparam monochrome true
		[-> ActorDeltaState: loadActorDeltasForActor
		Database --> ActorDeltaState: connection
		ActorDeltaState -> ActorDeltaState: Shim every loaded object
		ActorDeltaState ->]: loadActorDeltasForActorSuccess
		@enduml
	\end{plantuml}
	\caption{Szereplő első betöltődésekor az összes deltáit is betöltjük}
	\label{fig:global-state-bootstrap}
\end{figure}

