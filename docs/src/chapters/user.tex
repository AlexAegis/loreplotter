% !TeX root = ../thesis.tex
\chapter{Felhasználói dokumentáció}
\label{ch:user}

Az alkalmazás sem telepítést sem regisztrációt nem igényel, teljes egészében böngészőből használható. Minden alkalmazáson belüli tevékenység egy projekthez köthető melyeket a felhasználó kedve szerint hozhat létre, törölhet, módosíthat. Egyszerre több projektet is vezethetünk, a váltás közöttük azonnali.

\section{A Projekt}

Minden projekt \ref{fig:basic-project-structure} három fő elemből áll, ezek a név, a világ és később a szereplők. Új projekt létrehozásakor a projekt nevét és a világának adatait szükséges megadni.

\begin{figure}[h!]
	\centering
	\begin{forest}
		forked edges,
		for tree={edge+={-Latex}},
		[Projekt
			[Név]
			[Világ [Név] [Méret]]
			[Szereplők [\dots]]
		]
	\end{forest}
	\caption{
		Projektstruktúra}
	\label{fig:basic-project-structure}
\end{figure}

\begin{figure}[h!]
	\centering
	\includegraphics[width=0.50\textwidth,scale=1]{create-project-button}
	\caption{
		Új projektet létrehozni a jobb felső sarokban található menü legfelső elemével tehetjük meg}
	\label{fig:create-project-button}
\end{figure}

Új projekt létrehozásakor a projekt készítő dialógust kapjuk melyen lehetőséget kapunk a projektünknek nevet adni és a világbeállításokat megadni, mint a bolygó neve és mérete ami alatt a bolygó sugarát értjük kilóméterben megadva. Valamint egy tetszőleges fekete-fehér magasságtérképet is megadhatunk amivel inicializálhatjuk a bolygó felületét. A vízmagasság alapértelmezetten 40\%-os szürkeségnél van.


\begin{figure}[h!]
	\centering
	\includegraphics[width=0.80\textwidth,scale=1]{create-project-form}
	\caption{
		Új projekt dialógus}
	\label{fig:create-project-form}
\end{figure}


\begin{table}[H]
	\centering
	\begin{tabular}{ | m{0.25\textwidth} | m{0.10\textwidth} | m{0.55\textwidth} | }
		\hline
		\textbf{Mező} & \textbf{Típus} & \textbf{Leírás} \\
		\hline \hline
		\emph{Név} & Szöveg & Kötelező. Egyedi. A projekt azonosítója. \\
		\hline
		\emph{Bolygó név} & Szöveg & Kötelező. Alapértelmezett értéke: "Earth" \\
		\hline
		\emph{Bolygó méret} & Szám & Kötelező. Alapértelmezett értéke: "6371". (A föld sugara kilóméterben)  \\
		\hline
		\emph{Magasságtérkép} & Kép & Opcionális. Egy tetszőleges kezdeti magasságtérkép  \\
		\hline
	\end{tabular}
	\caption{Új projekt sablon}
	\label{tab:create-project-form}
\end{table}

A 'Create' gombra kattintva azonnal megnyílik az elkészített projekt.

\section{Kezelőfelület} \label{section:ui}

\begin{figure}[h!]
	\centering
	\includegraphics[width=1.0\textwidth,scale=1]{full-screenshot-open-sidebar}
	\caption{
		A kezelőfelület}
	\label{fig:full-screenshot-open-sidebar}
\end{figure}

A kezelőfelület az alábbi elemeket tartalmazza:

\begin{itemize}
	\item Bal oldalt a szereplőtár amibel készíteni és válogatni tudunk a szereplőink között. \ref{section:ui-actors}
	\item Alul az idővonal látható ahol az éppen megtekintett időpontot tudjuk megtekintni/változtatni. Az események idejét szerkeszteni.  \ref{section:ui-timeline}
\end{itemize}

\subsection{Szereplőtár} \label{section:ui-actors}

A szereplőtár legfelső elemével lehet újabb szereplőket hozzáadni a színtérhez. A gombot fogd-és-vidd módon lehet a bolygó valamely pontjára ráhúzni, ami majd a karakter kiindulási pontja lesz. Ez akármikor megváltoztatható.

Ez alatt az elkészített karakterek listája van, mellette az összes karakter számával. Egy karakter itt az alábbi információt tartalmazza:

\begin{itemize}
	\item Név (Akkumulált a jelenlegi időpillanatra)
	\item Szín (Akkumulált a jelenlegi időpillanatra)
	\item Összes esemény száma
\end{itemize}

\subsection{Idővonal} \label{section:ui-timeline}

A idővonalat elsősorban három elemre tudjuk bontani:

\subsubsection{A Kurzor}

A vonalzón helyezkedik el. Mindig a jelenlegi időpontra mutat, melyet a kurzoron, fogd-és-vidd módszerrel megváltoztathatunk. Egy fölötte megjelenő buborék a pontos időt mutatja másodperc pontossággal.

\paragraph{A Vonalzó}

Beosztásai napokban értendőek, a kisebb fogai pedig óráknak. A vonalzót fogd és vidd módon tudjuk az időben előre vagy hátra eltolni. Az egérgörgővel pedig az aktív idősávot tudjuk kibővíteni, szűkíteni.

\paragraph{A Karakter Sávok}

Minden karakter saját sávval rendelkezik. Bal oldalt a sávhoz tartozó karakter akkumulált nevét lehet látni. A sávon belül pedig a blokkját ami eseményekre van felosztva.

Egy blokk az alábbi interakciós lehetőségeket nyújtja:
\begin{enumerate}
	\item Egy eseményre kattintva a kurzor arra az időpontra ugrik és az esemény kiválasztásra kerül. Kiválasztás után lehetőség van kitörölni az adott eseményt egy megerősítést követően.
	\item Egy esemény fogd-és-vidd módon áthelyezhetünk bárhova az idővonalon. Figyeljünk oda, hogy ha a jövőben egy esemény függ ettől, és későbbi időpontra helyezzük, akkor ez a függőség sérül.
	\item A blokk egészét is mozgathatjuk fogd és vidd módon, ezzel az összes esemény párhuzamosan fog elmozdulni.
\end{enumerate}

Az idővonalon egyszerre 4 sáv fér el. A többi sávot görgetéssel, függőleges pásztázással, vagy a görgősávval érjük el.

\subsection{A Színtér} \label{section:ui-scene}

A kezelőfelület közepén helyezkedik el a 3D színtér és az azt közvetlen kezelő eszközök. A színtér fő eleme a bolygó és rajta a szereplők. A bolygó fogd-és-vidd módon forgatható, az egérgörgővel pedig közelíthető/távolítható.

A bolygó felületén lévő szereplők helyét, a szereplőt reprezentáló színes gömb fogd-és-vidd módon történő áthelyezésével lehet megtenni. Ennek a viselkedése nagyban függ a jelenlegi időtől és a szereplő azon eseményeitől mely a jelenlegi időponton, vagy a körül helyezkednek el. Minden szereplő rendelkezik egy maximum elérhető sebességgel. Hogy ez be legyen tartva, egy szereplő áthelyezésekor, csak olyan távolságba engedett az áthelyezés, amivel még el lehet érni az előző és a következő esemény helyét. Egy esemény helye áthelyezhető ha áthelyezéskor az az esemény ideje van kiválasztva, egyéb esetekben új esemény jön létre.


\subsubsection{Eszközök} \label{section:ui-utilities}

A színtér 3 sarkában 3 eszköztár látható.

\subsubsection{Az interakciós eszköztár}

\begin{figure}[h!]
	\centering
	\includegraphics[width=0.60\textwidth,scale=1]{tool-interaction}
	\caption{Interakciós eszköztár, a színtér bal felső részén.}
	\label{fig:tool-interaction}
\end{figure}

A ceruza ikonnal ellátott gombbal rajzolás módba válthatunk, a mellette lévő két csúszka pedig az ecset paraméterei mint magasság és méret. A bolygó felületét pászázva azt megváltoztathatjuk. Alacsony magassággal tengereket is rajzolhatunk.

A tengely ikonnal ellátott gombbal az alap, színtér mozgatáshoz szükséges módba térhetünk vissza. A mellette lévő csúszka a szereplőket reprezentáló gömbök méretét állítja. A gömbök mérete ettől, és a zoom mértékétől is függ.


\subsubsection{Az fény eszköztár}

\begin{figure}[h!]
	\centering
	\includegraphics[width=0.20\textwidth,scale=1]{tool-light-manual-dark}
	\caption{A fény eszköztár manuális módban, sötétre állítva. A színtér bal alsó részén.}
	\label{fig:tool-light-manual-dark}
\end{figure}

A kör alakú gombbal manuális és automata módok között válthatunk. A jobb oldali gomb csak manuális módban látszik. Ezzel tudunk váltani a permanens világos és sötét mód között.

\paragraph{Világos mód}

Világos módban a kezelőfelület teljes egésze fehérre vált, ahogy színtér háttere is. A színtérben a nap eltűnik és a bolygó egésze megvilágításra kerül.

\paragraph{Sötét mód}

Sötét módban a kezelőfelület teljes egésze sötétre vált, ahogy a színtér háttere is. A színtérben a nap csak ebben a módban van jelen, és a bolygót érő legerősebb fényforrássá válik.

\paragraph{Automata mód}

Automata módban alapértelmezetten a sötét mód aktív de túl közeli zoomnál, vagy akkor ha az idő túl gyorsan telik automatikusan a világos mód kapcsol be.


\subsubsection{Az idő eszköztár}

\begin{figure}[h!]
	\centering
	\includegraphics[width=0.40\textwidth,scale=1]{tool-time-dark}
	\caption{Az idő eszköztár, a színtér jobb alsó részén}
	\label{fig:tool-time-dark}
\end{figure}

Az idő eszköztár fő eleme a lejátszás gomb. Erre kattintva a csúszkával vagy a gombokkal beállított sebességgel megkezdődik a lejátszás. A sebesség egysége egy percet jelent minden valódi másodpercre. A lejátszás megkezdésével a színtér szereplői az eseményeik pozíciói között fognak mozogni.

Az alábbi billentyűkkel is vezérelhető a lejátszás:

\begin{table}[H]
	\centering
	\begin{tabular}{ | m{0.15\textwidth} | m{0.55\textwidth} | }
		\hline
		\textbf{Billentyű} & \textbf{Hatás}\\
		\hline \hline
		\emph{Szóköz} & A lejátszás megkezdése, szünetelése. \\
		\hline
		\emph{Számok 0-9} & A kettő adott hatványával való lejátszás. \\
		\hline
		\emph{Jobbra nyíl} & A lejátszás irányát pozitívra állítja.  \\
		\hline
		\emph{Balra nyíl} & A lejátszás irányát negatívra állítja.  \\
		\hline
		\emph{Felfele nyíl} & A lejátszást gyorsítja 60 egységgel. \\
		\hline
		\emph{Lefele nyíl} & A lejátszást lassítja 60 egységgel.  \\
		\hline
	\end{tabular}
	\caption{Idő vezérlése gombokkal}
	\label{tab:tool-time-control-keys}
\end{table}

\subsection{A Szereplők} \label{section:ui-actors}

A színtér szereplői kattintással, vagy a bal oldali menüből is kiválaszthatóak. Kiválasztást követően egy kis ablak jelenik meg a szereplő színtérbeli pozíciójánál mutatva az összes akkumulált adatát.

\begin{figure}[h!]
	\centering
	\includegraphics[width=0.40\textwidth,scale=1]{actor-popup}
	\caption{Az szereplő ablaka a jelenlegi adataival}
	\label{fig:actor-popup}
\end{figure}

A szerkesztés gombra kattintva pedig az adott időpillanat, adott helyzetén szerkeszthetjük a szereplő adatait. Minden változtatás a fent megjelenő időtől lesz érvényes.


\begin{table}[H]
	\centering
	\begin{tabular}{ | m{0.25\textwidth} | m{0.10\textwidth} | m{0.55\textwidth} | }
		\hline
		\textbf{Mező} & \textbf{Típus} & \textbf{Leírás} \\
		\hline \hline
		\emph{Dátum} & Szöveg & Megváltoztatható a szerkesztés dátuma. \\
		\hline
		\emph{Idő} & Szöveg & Megváltoztatható a szerkesztés időpontja. \\
		\hline
		\emph{Név} & Szöveg & A szereplő neve.  \\
		\hline
		\emph{Maximális sebesség} & Szám & A szereplő sebessége kilóméter per órában. \\
		\hline
		\emph{Szín} & Szín & A szereplő színe. \ref{fig:actor-edit-dialog-color}  \\
		\hline
	\end{tabular}
	\caption{Szereplő alap adatai}
	\label{tab:create-project-form}
\end{table}

\begin{figure}[h!]
	\centering
	\includegraphics[width=0.70\textwidth,scale=1]{actor-edit-dialog}
	\caption{Az szereplő szerkesztés abalaka}
	\label{fig:actor-edit-dialog}
\end{figure}

A szereplőnek lehetnek már meglévő, akkumulált tulajdonságai is mikor az ablakot megnyitjuk. Ezeket értékét megváltozathatjuk, de kitörölni itt nem tudjuk, ugyanis nem itt születtek. De "elfelejthetjük" őket aminek hatására az adott időpillanat után már nem lesz jelen a tulajdonság.

Lehetőség van új tulajdonságok felvitelére is, itt jelennek meg az adott eseményen született tulajdonságok. Ezek törölhetőek mert nem akkumulált adatok. \ref{fig:actor-edit-dialog-edit-forget}

\begin{figure}[h!]
	\centering
	\includegraphics[width=0.70\textwidth,scale=1]{actor-edit-dialog-color}
	\caption{Az szereplő színének megváltoztatása}
	\label{fig:actor-edit-dialog-color}
\end{figure}

\begin{figure}[h!]
	\centering
	\includegraphics[width=0.70\textwidth,scale=1]{actor-edit-dialog-edit-forget}
	\caption{Az szereplő meglévő tulajdonságainak elvesztése, új hozzádása}
	\label{fig:actor-edit-dialog-edit-forget}
\end{figure}


