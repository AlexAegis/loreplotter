\chapter{Tesztelés}
\label{ch:intro}

A tesztelés inkrementálisan, fejlesztés közben manuálisan zajlott.

Az alkalmazáson az alábbi manuális teszt sztorikat érdemes végigjátszani, hogy maximális lefedést érjünk el:

\begin{table}[H]
	\centering
	\begin{tabular}{ | m{0.20\textwidth} | m{0.80\textwidth} | }
		\hline
		\emph{Előfeltétel} & Az alkalmazás elindult  \\
		\hline
		\emph{Eredmény} & Új projekt kerül létrehozásra és megnyitásra  \\
		\hline
		\emph{Sikertelen Eredmény} & Az előzőleg kiválasztott projekt marad megnyitva. Új adat nem kerül mentésre.  \\
		\hline
		\emph{Események} &

		\begin{enumerate}
			\item A felhasználó a jobb fent található lenyíló menüben rákattint a 'Create' gombra
			\item Megjelenik a projekt készítő form
			\item A kötelező név mezőt kitölti
			\begin{enumerate}
				\item Foglalt név esetény a form invalid, erről hibaüzenet jelenik meg
				\item Üres név esetény a form invalid, erről hibaüzenet jelenik meg
			\end{enumerate}
			\item Opcionálisan átszerkeszti a bolygó adatait is bármilyen nem üres értékre (sugár minimum 500)
			\item A 'Create' gombra kattint.
			\begin{enumerate}
				\item A 'Cancel' gombra, vagy a dialógusablakon kívülre kattintva a folyamat véget ér és sikertelen eredménnyel zár.
			\end{enumerate}
		\end{enumerate}

		\\
		\hline
	\end{tabular}
	\caption{Sztori: Projekt készítése textúra nélkül}
	\label{tab:story-project-create}
\end{table}

\begin{table}[H]
	\centering
	\begin{tabular}{ | m{0.20\textwidth} | m{0.80\textwidth} | }
		\hline
		\emph{Előfeltétel} & Az alkalmazás elindult  \\
		\hline
		\emph{Eredmény} & Új projekt kerül létrehozásra és megnyitásra textúrával  \\
		\hline
		\emph{Sikertelen Eredmény} & Az előzőleg kiválasztott projekt marad megnyitva. Új adat nem kerül mentésre.  \\
		\hline
		\emph{Események} &

		\begin{enumerate}
			\item A felhasználó a jobb fent található lenyíló menüben rákattint a 'Create' gombra
			\item Megjelenik a projekt készítő form
			\item A kötelező név mezőt kitölti
			\begin{enumerate}
				\item Foglalt név esetény a form invalid, erről hibaüzenet jelenik meg
				\item Üres név esetény a form invalid, erről hibaüzenet jelenik meg
			\end{enumerate}
			\item Opcionálisan átszerkeszti a bolygó adatait is bármilyen nem üres értékre (sugár minimum 500)
			\item A fájlfeltöltés eszközzel feltölt egy tetszőleges textúrát magasságtérképnek
			\item A 'Create' gombra kattint
			\begin{enumerate}
				\item A 'Cancel' gombra, vagy a dialógusablakon kívülre kattintva a folyamat véget ér és sikertelen eredménnyel zár.
			\end{enumerate}
		\end{enumerate}

		\\
		\hline
	\end{tabular}
	\caption{Sztori: Projekt készítése textúrával}
	\label{tab:story-project-create-texture}
\end{table}

\begin{table}[H]
	\centering
	\begin{tabular}{ | m{0.20\textwidth} | m{0.80\textwidth} | }
		\hline
		\emph{Előfeltétel} & Egy projekt meg van nyitva  \\
		\hline
		\emph{Eredmény} & A projekt adatai megváltoznak. Ha töltött fel textúrát akkor a bolygó magasságtérképe is megváltozik    \\
		\hline
		\emph{Események} &

		\begin{enumerate}
			\item A felhasználó a jobb fent található projekt gombra rákattint
			\item Megjelenik a projekt szerkesztő form
			\item Módosításokat hajt végre
			\begin{enumerate}
				\item Foglalt név esetény a form invalid, erről hibaüzenet jelenik meg
				\item Üres név esetény a form invalid, erről hibaüzenet jelenik meg
				\item Üres bolygó név esetény a form invalid, erről hibaüzenet jelenik meg
				\item 500-nál kisebb bolygó sugár esetény a form invalid, erről hibaüzenet jelenik meg
				\item Ha a bolygó sugár nagyobb mint ami esetén bármely szereplő már ne tudná megtenni egy útját, akkor a form invalid, erről hibaüzenet jelenik meg
			\end{enumerate}
			\item Opcionálisan a fájlfeltöltés eszközzel feltölt egy tetszőleges textúrát magasságtérképnek
			\item A 'Create' gombra kattint
			\begin{enumerate}
				\item A 'Cancel' gombra, vagy a dialógusablakon kívülre kattintva a folyamat véget ér és sikertelen eredménnyel zár.
			\end{enumerate}
		\end{enumerate}

		\\
		\hline
	\end{tabular}
	\caption{Sztori: Projekt Szerkesztése}
	\label{tab:story-project-edit}
\end{table}


\begin{table}[H]
	\centering
	\begin{tabular}{ | m{0.20\textwidth} | m{0.80\textwidth} | }
		\hline
		\emph{Előfeltétel} & Egy projekt meg van nyitva  \\
		\hline
		\emph{Eredmény} & A bolygó felülete az alapértelmezett föld felületre változik.    \\
		\hline
		\emph{Események} &

		\begin{enumerate}
			\item A felhasználó a jobb fent található projekt gombra rákattint
			\item Megjelenik a projekt szerkesztő form
			\item Módosításokat hajt végre
			\begin{enumerate}
				\item Foglalt név esetény a form invalid, erről hibaüzenet jelenik meg
				\item Üres név esetény a form invalid, erről hibaüzenet jelenik meg
				\item Üres bolygó név esetény a form invalid, erről hibaüzenet jelenik meg
				\item 500-nál kisebb bolygó sugár esetény a form invalid, erről hibaüzenet jelenik meg
				\item Ha a bolygó sugár nagyobb mint ami esetén bármely szereplő már ne tudná megtenni egy útját, akkor a form invalid, erről hibaüzenet jelenik meg
			\end{enumerate}
			\item Opcionálisan a fájlfeltöltés eszközzel feltölt egy tetszőleges textúrát magasságtérképnek
			\item A 'Create' gombra kattint
			\begin{enumerate}
				\item A 'Cancel' gombra, vagy a dialógusablakon kívülre kattintva a folyamat véget ér és sikertelen eredménnyel zár.
			\end{enumerate}
		\end{enumerate}

		\\
		\hline
	\end{tabular}
	\caption{Sztori: Alapértelmezett Föld textúra visszaállítása}
	\label{tab:story-project-edit-earth-texture}
\end{table}
