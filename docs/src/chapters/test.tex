\chapter{Tesztelés}
\label{ch:intro}

A tesztelés inkrementálisan, fejlesztés közben manuálisan zajlott.

Az alkalmazáson az alábbi manuális teszt sztorikat érdemes végigjátszani, hogy maximális lefedést érjünk el:

\section{Projekt interakciók}

\begin{table}[H]
	\centering
	\begin{tabular}{ | m{0.14\textwidth} | m{0.86\textwidth} | }
		\hline
		\emph{Előfeltétel} & Az alkalmazás elindult.  \\
		\hline
		\emph{Kiváltja} & A felhasználó a jobb fent található lenyíló menüben rákattint a 'Create' gombra. \\
		\hline
		\emph{Eredmény} & Új projekt kerül létrehozásra és megnyitásra.  \\
		\hline
		\emph{Sikertelen Eredmény} & Az előzőleg kiválasztott projekt marad megnyitva. Új adat nem kerül mentésre.  \\
		\hline
		\hline
		\emph{Események} &

		\begin{enumerate}[itemsep=-1ex]
			\item Megjelenik a projekt készítő form.
			\item A kötelező név mezőt kitölti.
			\begin{enumerate}[itemsep=-1ex]
				\item Foglalt név esetén a form invalid, erről hibaüzenet jelenik meg.
				\item Üres név esetén a form invalid, erről hibaüzenet jelenik meg.
			\end{enumerate}
			\item Opcionálisan átszerkeszti a bolygó adatait is bármilyen nem üres értékre. (sugár minimum 500)
			\item A 'Create' gombra kattint, vagy lenyomja az 'Enter' billentyűt, ekkor a folyamat sikeres eseménnyel zár
			\begin{enumerate}[itemsep=-1ex]
				\item A 'Cancel' gombra, vagy a dialógusablakon kívülre kattintva a folyamat véget ér és sikertelen eredménnyel zár.
			\end{enumerate}
		\end{enumerate}

		\\
		\hline
	\end{tabular}
	\caption{Sztori: Projekt készítése textúra nélkül.}
	\label{tab:story-project-create}
\end{table}

\begin{table}[H]
	\centering
	\begin{tabular}{ | m{0.14\textwidth} | m{0.86\textwidth} | }
		\hline
		\emph{Előfeltétel} & Az alkalmazás elindult.  \\
		\hline
		\emph{Kiváltja} & A felhasználó a jobb fent található lenyíló menüben rákattint a 'Create' gombra. \\
		\hline
		\emph{Eredmény} & Új projekt kerül létrehozásra és megnyitásra textúrával.  \\
		\hline
		\emph{Sikertelen Eredmény} & Az előzőleg kiválasztott projekt marad megnyitva. Új adat nem kerül mentésre.  \\
		\hline
		\hline
		\emph{Események} &

		\begin{enumerate}[itemsep=-1ex]
			\item Megjelenik a projekt készítő form.
			\item A kötelező név mezőt kitölti.
			\begin{enumerate}[itemsep=-1ex]
				\item Foglalt név esetén a form invalid, erről hibaüzenet jelenik meg.
				\item Üres név esetén a form invalid, erről hibaüzenet jelenik meg.
			\end{enumerate}
			\item Opcionálisan átszerkeszti a bolygó adatait is bármilyen nem üres értékre. (sugár minimum 500)
			\item A fájlfeltöltés eszközzel feltölt egy tetszőleges textúrát magasságtérképnek.
			\item A 'Create' gombra kattint, vagy lenyomja az 'Enter' billentyűt, ekkor a folyamat sikeres eseménnyel zár
			\begin{enumerate}[itemsep=-1ex]
				\item A 'Cancel' gombra, vagy a dialógusablakon kívülre kattintva a folyamat véget ér és sikertelen eredménnyel zár.
			\end{enumerate}
		\end{enumerate}

		\\
		\hline
	\end{tabular}
	\caption{Sztori: Projekt készítése textúrával.}
	\label{tab:story-project-create-texture}
\end{table}

\begin{table}[H]
	\centering
	\begin{tabular}{ | m{0.14\textwidth} | m{0.86\textwidth} | }
		\hline
		\emph{Előfeltétel} & Egy projekt meg van nyitva.  \\
		\hline
		\emph{Kiváltja} & A felhasználó a jobb fent található projekt gombra kattint. \\
		\hline
		\emph{Eredmény} & A projekt adatai megváltoznak. Ha töltött fel textúrát akkor a bolygó magasságtérképe is megváltozik.    \\
		\hline
		\emph{Sikertelen Eredmény} & Nem történik változás.  \\
		\hline
		\hline
		\emph{Események} &

		\begin{enumerate}[itemsep=-1ex]
			\item Megjelenik a projekt szerkesztő form.
			\item Módosításokat hajt végre.
			\begin{enumerate}[itemsep=-1ex]
				\item Foglalt név esetén a form invalid, erről hibaüzenet jelenik meg.
				\item Üres név esetén a form invalid, erről hibaüzenet jelenik meg.
				\item Üres bolygó név esetén a form invalid, erről hibaüzenet jelenik meg.
				\item 500-nál kisebb bolygó sugár esetén a form invalid, erről hibaüzenet jelenik meg.
				\item Ha a bolygó sugár nagyobb mint ami esetén bármely szereplő már ne tudná megtenni egy útját, akkor a form invalid, erről hibaüzenet jelenik meg.
			\end{enumerate}
			\item Opcionálisan a fájlfeltöltés eszközzel feltölt egy tetszőleges textúrát magasságtérképnek.
			\item A 'Create' gombra kattint, vagy lenyomja az 'Enter' billentyűt, ekkor a folyamat sikeres eseménnyel zár
			\begin{enumerate}[itemsep=-1ex]
				\item A 'Cancel' gombra, vagy a dialógusablakon kívülre kattintva a folyamat véget ér és sikertelen eredménnyel zár.
			\end{enumerate}
		\end{enumerate}

		\\
		\hline
	\end{tabular}
	\caption{Sztori: Projekt Szerkesztése.}
	\label{tab:story-project-edit}
\end{table}


\begin{table}[H]
	\centering
	\begin{tabular}{ | m{0.14\textwidth} | m{0.86\textwidth} | }
		\hline
		\emph{Előfeltétel} & Egy projekt meg van nyitva.  \\
		\hline
		\emph{Kiváltja} & A felhasználó a jobb fent található projekt gombra kattint. \\
		\hline
		\emph{Eredmény} & A bolygó felülete az alapértelmezett föld felületre változik. \\
		\hline
		\emph{Sikertelen Eredmény} & Nem történik változás.  \\
		\hline
		\hline
		\emph{Események} &

		\begin{enumerate}[itemsep=-1ex]
			\item Megjelenik a projekt szerkesztő form.
			\item Módosításokat hajt végre.
			\begin{enumerate}[itemsep=-1ex]
				\item Foglalt név esetén a form invalid, erről hibaüzenet jelenik meg.
				\item Üres név esetén a form invalid, erről hibaüzenet jelenik meg.
				\item Üres bolygó név esetén a form invalid, erről hibaüzenet jelenik meg.
				\item 500-nál kisebb bolygó sugár esetén a form invalid, erről hibaüzenet jelenik meg.
				\item Ha a bolygó sugár nagyobb mint ami esetén bármely szereplő már ne tudná megtenni egy útját, akkor a form invalid, erről hibaüzenet jelenik meg.
			\end{enumerate}
			\item Opcionálisan a fájlfeltöltés eszközzel feltölt egy tetszőleges textúrát magasságtérképnek.
			\item A 'Create' gombra kattint, vagy lenyomja az 'Enter' billentyűt, ekkor a folyamat sikeres eseménnyel zár
			\begin{enumerate}[itemsep=-1ex]
				\item A 'Cancel' gombra, vagy a dialógusablakon kívülre kattintva a folyamat véget ér és sikertelen eredménnyel zár.
			\end{enumerate}
		\end{enumerate}

		\\
		\hline
	\end{tabular}
	\caption{Sztori: Alapértelmezett Föld textúra visszaállítása.}
	\label{tab:story-project-edit-earth-texture}
\end{table}

\section{Szereplő interakciók}

\begin{table}[H]
	\centering
	\begin{tabular}{ | m{0.14\textwidth} | m{0.86\textwidth} | }
		\hline
		\emph{Előfeltétel} & Egy projekt meg van nyitva, az oldalsáv látszik.  \\
		\hline
		\emph{Eredmény} & A bolygó felületén az egér alatt a szereplő megjelnik. Az idővonalon a szereplő sávja megjelenik 1 eseménnyel.   \\
		\hline
		\emph{Sikertelen Eredmény} & Nem jelenik meg új szereplő.  \\
		\hline
		\hline
		\emph{Események} &

		\begin{enumerate}[itemsep=-1ex]
			\item A felhasználó az oldalsáv 'Actor' gombját fogd és vidd módon ráhúzza a bolygó egy pontjára.
			\item Vagy rákattint az 'Actor' gombra majd a bolygó egy pontjára.
			\item Ha nem a bolygó egy pontjára kattint vagy dob, akkor sikertelen eredménnyel zár.
		\end{enumerate}

		\\
		\hline
	\end{tabular}
	\caption{Sztori: Szereplő készítése}
	\label{tab:story-actor-create}
\end{table}

\begin{table}[H]
	\centering
	\begin{tabular}{ | m{0.14\textwidth} | m{0.86\textwidth} | }
		\hline
		\emph{Előfeltétel} & Egy szereplő legalább létre van hozva.  \\
		\hline
		\emph{Eredmény} & A színtéren a szereplő mellett megjelenik az információs buborék.   \\
		\hline
		\hline
		\emph{Események} &

		\begin{enumerate}[itemsep=-1ex]
			\item A felhasználó az oldalsávban rákattint egy szereplőnek egy gombjára.
			\item Vagy a színtérben rákattint egy szereplő objektumára.
		\end{enumerate}
		\\
		\hline
	\end{tabular}
	\caption{Sztori: Szereplő kiválasztása}
	\label{tab:story-actor-create}
\end{table}

\begin{table}[H]
	\centering
	\begin{tabular}{ | m{0.14\textwidth} | m{0.86\textwidth} | }
		\hline
		\emph{Előfeltétel} & Egy szereplő legalább létre van hozva. Ki van választva, a kurzor pontosan egy eseményén áll a szereplőnek. Szerkesztés dialógusa meg van nyitva. Szerkesztés közben az időpontot nem változtatja meg. \\
		\hline
		\emph{Eredmény} & Összesen ugyanannyi eseménye marad mint volt, a szerkesztett időponton lévő adatai megváltoznak. Csak azok amiknek a mezőit nem hagytuk üresen.    \\
		\hline
		\emph{Sikertelen Eredmény} & Nem történik változtatás.  \\
		\hline
		\hline
		\emph{Események} &

		\begin{enumerate}[itemsep=-1ex]
			\item A felhasználó kitölti a megváltoztatandó adatokat.
			\item Lehetősége van kitörölni az eseményen korábban elkészített tulajdonságokat.
			\item A mentés gombra kattint vagy lenyomja az 'Enter' billentyűt, ekkor a folyamat sikeres eseménnyel zár. Ha más módon zárja be az ablakot a folyamaz sikertelen eredménnyel zár.
		\end{enumerate}
		\\
		\hline
	\end{tabular}
	\caption{Sztori: Szereplő szerkesztése meglévő eseményen.}
	\label{tab:story-actor-edit-on-event}
\end{table}

\begin{table}[H]
	\centering
	\begin{tabular}{ | m{0.14\textwidth} | m{0.86\textwidth} | }
		\hline
		\emph{Előfeltétel} & Egy szereplő legalább létre van hozva. Ki van választva, a kurzor nem egy eseményén áll a szereplőnek. Szerkesztés dialógusa meg van nyitva. \\
		\hline
		\emph{Eredmény} & Összesen eggyel több eseménye lesz mint volt, a szerkesztett időponton lévő adatai létrejönnek, és csak azok amik ki lettek töltve. A pozíciója nem változik meg mint ahol lett volna a végleges időponton. \\
		\hline
		\emph{Sikertelen Eredmény} & Nem történik változtatás.  \\
		\hline
		\hline
		\emph{Események} &

		\begin{enumerate}[itemsep=-1ex]
			\item A felhasználó kitölti a megváltoztatandó adatokat.
			\item Lehetősége van kitörölni az eseményen korábban elkészített tulajdonságokat.
			\item A mentés gombra kattint vagy lenyomja az 'Enter' billentyűt, ekkor a folyamat sikeres eseménnyel zár. Ha más módon zárja be az ablakot a folyamaz sikertelen eredménnyel zár.
		\end{enumerate}
		\\
		\hline
	\end{tabular}
	\caption{Sztori: Szereplő szerkesztése nem meglévő eseményen állva.}
	\label{tab:story-actor-edit-not-on-event}
\end{table}




\section{Idővonal interakciók}

\begin{table}[H]
	\centering
	\begin{tabular}{ | m{0.14\textwidth} | m{0.86\textwidth} | }
		\hline
		\emph{Előfeltétel} & Egy projekt meg van nyitva. \\
		\hline
		\emph{Eredmény} & A kurzor annyit csúszik előre vagy hátra amennyit a felhasználó mozdított az egerén. \\
		\hline
		\hline
		\emph{Események} &
		\begin{enumerate}[itemsep=-1ex]
			\item A felhasználó rákattint a kurzor egy pontjára és lenyomva tartja az egeret.
			\item Az egeret egy tetszőleges mértékben balra vagy jobbra elmozdítja.
			\item Elengedéskor a folyamat sikeres eredménnyel zár.
		\end{enumerate}
		\\
		\hline
	\end{tabular}
	\caption{Sztori: A kurzor kézi pásztázása.}
	\label{tab:story-timeline-manual-pan-cursor}
\end{table}

\begin{table}[H]
	\centering
	\begin{tabular}{ | m{0.14\textwidth} | m{0.86\textwidth} | }
		\hline
		\emph{Előfeltétel} & Egy projekt meg van nyitva. A lejátszás szünetel. \\
		\hline
		\emph{Eredmény} & A kurzor a megadott időpontra ugrik. Ha szükséges (Nem a jelenlegi időablak belső 70\%-n van a cél időpont) akkor az időablak is odaugrik, hogy a kurzor látszódjon.  \\
		\hline
		\emph{Sikertelen Eredmény} & Nem történik változtatás.  \\
		\hline
		\hline
		\emph{Események} &
		\begin{enumerate}[itemsep=-1ex]
			\item A felhasználó rákattint a kurzoron a dátumra.
			\item Majd átszerkeszti az időt egy tetszőleges, a moment csomag által értelmezhető formátumra. (Nagyon megengedő formátumok terén)
			\item Ha valami más olyan esemény ami a kurzor idejének megváltozását idézi elő bekövetkezik, akkor az megszakítja a folyamatot és az sikertelen eredménnyel zár.
			\item Ha nem ismeri fel a formátumot a folyamat sikertelen eredménnyel zár.
			\item Ha felismeri a folyamaz sikeres eredménnyel zár.
		\end{enumerate}
		\\
		\hline
	\end{tabular}
	\caption{Sztori: A kurzor szövegesen megadott időpontra ugratása.}
	\label{tab:story-timeline-manual-jump-cursor}
\end{table}

\begin{table}[H]
	\centering
	\begin{tabular}{ | m{0.14\textwidth} | m{0.86\textwidth} | }
		\hline
		\emph{Előfeltétel} & Egy szereplő legalább létre van hozva, hogy lássuk az idővonalon a mozgását. \\
		\hline
		\emph{Eredmény} & Az időablak annyit csúszik előre vagy hátra amennyit a felhasználó mozdított az egerén. \\
		\hline
		\hline
		\emph{Események} &
		\begin{enumerate}[itemsep=-1ex]
			\item A felhasználó rákattint az idővonal vonalzójának egy pontjára és lenyomva tartja az egeret.
			\item Az egeret egy tetszőleges mértékben balra vagy jobbra elmozdítja.
			\item Elengedéskor a folyamat sikeres eredménnyel zár.
		\end{enumerate}
		\\
		\hline
	\end{tabular}
	\caption{Sztori: Az időablak kézi pásztázása.}
	\label{tab:story-timeline-manual-pan-timeframe}
\end{table}

\begin{table}[H]
	\centering
	\begin{tabular}{ | m{0.14\textwidth} | m{0.86\textwidth} | }
		\hline
		\emph{Előfeltétel} & Egy szereplő legalább létre van hozva, hogy lássuk az idővonalon a mozgását.\\
		\hline
		\emph{Eredmény} & Az időablak annyit ugrik előre vagy hátra, hogy a kurzor középen legyen. \\
		\hline
		\emph{Sikertelen Eredmény} & A kurzor helyzete megváltozik de az időablak nem.  \\
		\hline
		\hline
		\emph{Események} &

		\begin{enumerate}[itemsep=-1ex]
			\item A felhasználó rákattint az idővonal vonalzójának egy pontjára.
			\item Ha ez a pont az első vagy az utolsó 15\%-ában volt a teljes vonalzónak, akkor a folyamat sikeres eredménnyel zár. Ha nem, akkor sikertelennel.
		\end{enumerate}
		\\
		\hline
	\end{tabular}
	\caption{Sztori: Az időablak kézi ugratása.}
	\label{tab:story-timeline-manual-jump}
\end{table}

\begin{table}[H]
	\centering
	\begin{tabular}{ | m{0.14\textwidth} | m{0.86\textwidth} | }
		\hline
		\emph{Előfeltétel} & Egy szereplő legalább létre van hozva, hogy lássuk az idővonalon a mozgását.\\
		\hline
		\emph{Eredmény} & Az időablak zsugorodik és nyúlik azon időponttól el/időpont felé ami felett az egér áll. \\
		\hline
		\hline
		\emph{Események} &

		\begin{enumerate}[itemsep=-1ex]
			\item A felhasználó a vonalzó felett görgeti az egerét.
		\end{enumerate}
		\\
		\hline
	\end{tabular}
	\caption{Sztori: Az időablak nyújtása, zsugorítása.}
	\label{tab:story-timeline-zoom}
\end{table}


\begin{table}[H]
	\centering
	\begin{tabular}{ | m{0.14\textwidth} | m{0.86\textwidth} | }
		\hline
		\emph{Előfeltétel} & Egy szereplő legalább létre van hozva, egy eseménnyel.\\
		\hline
		\emph{Eredmény} & A szereplő egyetlen eseményének időpontja megváltozik. \\
		\hline
		\hline
		\emph{Események} &
		\begin{enumerate}[itemsep=-1ex]
			\item A felhasználó egy tetszőleges pozícióra húzza az eseményt.
		\end{enumerate}
		\\
		\hline
	\end{tabular}
	\caption{Sztori: Szereplő egyetlen eseményének mozgatása.}
	\label{tab:story-timeline-pan-actor-single-event}
\end{table}


\begin{table}[H]
	\centering
	\begin{tabular}{ | m{0.14\textwidth} | m{0.86\textwidth} | }
		\hline
		\emph{Előfeltétel} & Egy szereplő legalább létre van hozva, több mint egy eseménnyel.\\
		\hline
		\emph{Eredmény} & A szereplő eseményének ideje megváltozik. \\
		\hline
		\emph{Sikertelen Eredmény} & A szereplő eseménye nem változik meg, visszaugrik eredeti pozíciójára. \\
		\hline
		\hline
		\emph{Események} &
		\begin{enumerate}[itemsep=-1ex]
			\item A felhasználó egy tetszőleges időpontra húzza a szereplő kiválasztott eseményét, majd elengedi az egeret.
			\item Ha voltak környező eseményei, és a mozgatás célpontja kívül esett ezeken. És az eredetileg környező két esemény már nem tudná elérni egymást akkor a folyamat sikertelen eredménnyel zár.
			\item Ha a mozgatás célpontján, ha van előző esemény és az nem tudja elérni a mozgatott eseményt, vagy ha van következő esemény és a mozgatott esemény azt nem tudja elérni, akkor a folyamaz sikertelen eredménnyel zár.
			\item Egyébként a folyamat sikeres.
		\end{enumerate}
		\\
		\hline
	\end{tabular}
	\caption{Sztori: Több eseménnyel rendelkező szereplő egy eseményének mozgatása.}
	\label{tab:story-timeline-pan-actor-more-event}
\end{table}


\begin{table}[H]
	\centering
	\begin{tabular}{ | m{0.14\textwidth} | m{0.86\textwidth} | }
		\hline
		\emph{Előfeltétel} & Egy szereplő legalább létre van hozva, több mint egy eseménnyel.\\
		\hline
		\emph{Eredmény} & A szereplő összes eseményének ideje megváltozik egyenlő mértékben. \\
		\hline
		\hline
		\emph{Események} &
		\begin{enumerate}[itemsep=-1ex]
			\item A felhasználó egy tetszőleges időpontra húzza a szereplő blokkját, majd elengedi az egeret.
		\end{enumerate}
		\\
		\hline
	\end{tabular}
	\caption{Sztori: Több eseménnyel rendelkező szereplő összes eseményének mozgatása.}
	\label{tab:story-timeline-pan-actor-more-event-at-once}
\end{table}

\begin{table}[H]
	\centering
	\begin{tabular}{ | m{0.14\textwidth} | m{0.86\textwidth} | }
		\hline
		\emph{Előfeltétel} & Egy szereplő legalább létre van hozva, egy eseménnyel.\\
		\hline
		\emph{Eredmény} & A szereplő kitörlődik. \\
		\hline
		\emph{Sikertelen Eredmény} & A szereplő nem törlődik ki. \\
		\hline
		\hline
		\emph{Események} &
		\begin{enumerate}[itemsep=-1ex]
			\item A felhasználó kiválasztja a szereplő egyetlen eseményét.
			\item A megjelenő kuka gombra kattint.
			\item Megerősíti a döntését azzal, hogy a 'Confirm' gombra kattint vagy lenyomja az 'Enter' billentyűt. Ekkor a folyamat sikeres eredménnyel zár.
			\item  Megszakítja a döntését azzal, hogy a 'Cancel' gombra, vagy a dialóguson kívülre kattint vagy lenyomja az 'Esc' billentyűt. Ekkor a folyamat sikertelen eredménnyel zár.
		\end{enumerate}
		\\
		\hline
	\end{tabular}
	\caption{Sztori: Szereplő egyetlen eseményének törlése.}
	\label{tab:story-timeline-remove-actor-single-event}
\end{table}

\begin{table}[H]
	\centering
	\begin{tabular}{ | m{0.14\textwidth} | m{0.86\textwidth} | }
		\hline
		\emph{Előfeltétel} & Egy szereplő legalább létre van hozva, több eseménnyel.\\
		\hline
		\emph{Eredmény} & A szereplőnek egyel kevesebb eseménye lesz mint volt. \\
		\hline
		\emph{Sikertelen Eredmény} & Az esemény nem törlődik ki. \\
		\hline
		\hline
		\emph{Események} &
		\begin{enumerate}[itemsep=-1ex]
			\item A felhasználó kiválasztja a szereplő egy eseményét.
			\item A megjelenő kuka gombra kattint.
			\item Megerősíti a döntését azzal, hogy a 'Confirm' gombra kattint vagy lenyomja az 'Enter' billentyűt. Ekkor a folyamat sikeres eredménnyel zár.
			\item  Megszakítja a döntését azzal, hogy a 'Cancel' gombra, vagy a dialóguson kívülre kattint vagy lenyomja az 'Esc' billentyűt. Ekkor a folyamat sikertelen eredménnyel zár.
			\item Ha a kiválasztott eseménynek van előtte és utána lévő eseménye is. És ezek a törlést követően már nem érnék el egymást, úgy a törlés nem megengedett. A folyamat sikertelen eredménnyel zár.
		\end{enumerate}
		\\
		\hline
	\end{tabular}
	\caption{Sztori: Szereplő egy eseményének mozgatása.}
	\label{tab:story-timeline-remove-actor-more-event}
\end{table}

\begin{table}[H]
	\centering
	\begin{tabular}{ | m{0.14\textwidth} | m{0.86\textwidth} | }
		\hline
		\emph{Előfeltétel} & Egy szereplő legalább létre van hozva.\\
		\hline
		\emph{Eredmény} & A szereplőnek egyel több eseménye lesz mint volt. Ezen az eseményen ugyanaz a pozíció lesz rögzítve mint ahol amúgy is lett volna a szereplő lejátszáskor. \\
		\hline
		\emph{Sikertelen Eredmény} & Új esemény létrehozása helyett egy meglévő választódik ki. \\
		\hline
		\hline
		\emph{Események} &
		\begin{enumerate}[itemsep=-1ex]
			\item A felhasználó a szereplő sávján belül egy tetszőleges pontra kattint.
			\item Ha nem volt ott esemény, sikeres eredménnyel zárunk.
			\item Ha volt ott már esemény, sikertelen eredménnyel zárunk.
		\end{enumerate}
		\\
		\hline
	\end{tabular}
	\caption{Sztori: Üres esemény létrehozása.}
	\label{tab:story-timeline-create-event}
\end{table}

\section{Színtér interakciók}


\begin{table}[H]
	\centering
	\begin{tabular}{ | m{0.14\textwidth} | m{0.86\textwidth} | }
		\hline
		\emph{Előfeltétel} & Egy textúra be van töltve, hogy lássuk a bolygó mozgását. A színtér mozgatás módban van. \\
		\hline
		\emph{Eredmény} & A bolygó egy új pozícióból tekinthető meg. \\
		\hline
		\hline
		\emph{Események} &

		\begin{enumerate}[itemsep=-1ex]
			\item A felhasználó rákattint a bolygó egy pontjára ahol nincsen más objektum és az egeret húzva elforgatja a kamerát.
		\end{enumerate}
		\\
		\hline
	\end{tabular}
	\caption{Sztori: A kamera mozgatása.}
	\label{tab:story-planet-pan}
\end{table}


\begin{table}[H]
	\centering
	\begin{tabular}{ | m{0.14\textwidth} | m{0.86\textwidth} | }
		\hline
		\emph{Előfeltétel} & Egy projekt ki van választva. \\
		\hline
		\emph{Eredmény} & A bolygó egy új közelségből tekinthető meg \\
		\hline
		\hline
		\emph{Események} &

		\begin{enumerate}[itemsep=-1ex]
			\item A felhasználó az egerét a színtér felett tartva görget.
		\end{enumerate}
		\\
		\hline
	\end{tabular}
	\caption{Sztori: A kamera közelítése és távolítása.}
	\label{tab:story-planet-pan}
\end{table}

\begin{table}[H]
	\centering
	\begin{tabular}{ | m{0.14\textwidth} | m{0.86\textwidth} | }
		\hline
		\emph{Előfeltétel} & Egy projekt ki van választva. A színtér festés módban van. \\
		\hline
		\emph{Eredmény} & A bolygó új textúrája az adatbázisba mentésre kerül. \\
		\hline
		\hline
		\emph{Események} &

		\begin{enumerate}[itemsep=-1ex]
			\item A felhasználó kiválaszt egy tetszőleges ecsetméretet és erősséget.
			\item a bolygó felületére rajzol az egérrel.
		\end{enumerate}
		\\
		\hline
	\end{tabular}
	\caption{Sztori: A bolygó felületének módosítása.}
	\label{tab:story-terraforming}
\end{table}

\begin{table}[H]
	\centering
	\begin{tabular}{ | m{0.14\textwidth} | m{0.86\textwidth} | }
		\hline
		\emph{Előfeltétel} & Egy szereplő legalább létre van hozva. A színtéren elérhető helyen van. A kurzor a szereplő egy meglévő eseményén áll. \\
		\hline
		\emph{Eredmény} & Összesen ugyanannyi eseménye lesz mint volt, a szerkesztett időponton lévő adatai megmaradnak, csak a pozíciója változik meg. \\
		\hline
		\hline
		\emph{Események} &

		\begin{enumerate}[itemsep=-1ex]
			\item A felhasználó rákattint a szereplő objektumára és lenyomva tartja az egérgombot. Ekkor megjelennek az éppen kiválasztott esemény előtti és utáni eseményekhez tartozó távolság indikátorok (ha vannak) amik jelzik, hogy maximum mennyire távolodhatunk el tőlük a szerkesztés időpillanatában.
			\item A felhasználó egy tetszőleges hely felé húzza a kiválasztott objektumot. Az indikátorokon kívülre nem fogja engedni húzni az alkalmazás őket.
			\item Az egérgombot elengedve a pozíció rögzítésre kerül. A folyamat sikeres eredménnyel zár.
		\end{enumerate}
		\\
		\hline
	\end{tabular}
	\caption{Sztori: Szereplő mozgatása meglévő eseményen állva.}
	\label{tab:story-actor-move-on-event}
\end{table}


\begin{table}[H]
	\centering
	\begin{tabular}{ | m{0.14\textwidth} | m{0.86\textwidth} | }
		\hline
		\emph{Előfeltétel} & Egy szereplő legalább létre van hozva. A színtéren elérhető helyen van. A kurzor nem áll a szereplő egy meglévő eseményén. \\
		\hline
		\emph{Eredmény} & Összesen eggyel több eseménye lesz mint volt, a szerkesztett időponton egy új esemény, adatok nélkül jön létre, csak a pozíciója kerül elmentésre. \\
		\hline
		\hline
		\emph{Események} &

		\begin{enumerate}[itemsep=-1ex]
			\item A felhasználó rákattint a szereplő objektumára és lenyomva tartja az egérgombot. Ekkor megjelenik az előző esemény és a következő eseményekhez tartozó távolság indikátorai (ha vannak) amik jelzi, hogy maximum mennyire távolodhatunk el tőlük a szerkesztés időpillanatában.
			\item A felhasználó egy tetszőleges hely felé húzza a kiválasztott objektumot. Az indikátorokon kívülre nem fogja engedni húzni az alkalmazás őket.
			\item Az egérgombot elengedve a pozíció rögzítésre kerül. A folyamat sikeres eredménnyel zár.
		\end{enumerate}
		\\
		\hline
	\end{tabular}
	\caption{Sztori: Szereplő mozgatása nem egy meglévő eseményen állva.}
	\label{tab:story-actor-move-not-on-event}
\end{table}

\section{Idő interakciók}

\begin{table}[H]
	\centering
	\begin{tabular}{ | m{0.14\textwidth} | m{0.86\textwidth} | }
		\hline
		\emph{Előfeltétel} & Egy szereplő legalább létre van hozva. Több, eltérő helyen lévő eseménnyel. \\
		\hline
		\emph{Eredmény} & A kurzor pozíciója folyamatosan változik a beállított sebesség függvényében. Ha eléri az időablak utolsó vagy első 15\%-át akkor az időablak a kurzor irányába ugrik. A szereplő eseményei közötti átmeneteket megtekinthetjük. \\
		\hline
		\hline
		\emph{Események} &
		\begin{enumerate}[itemsep=-1ex]
			\item A felhasználó kiválaszt egy sebességet valamilyen módon:
			\begin{enumerate}[itemsep=-1ex]
				\item A szám billentyűkkel (0-9).
				\item A szám gombokkal a sebességcsúszka felett.
				\item A sebességcsúszkával.
				\item A fel és le nyilakkal.
			\end{enumerate}
			\item Akár a lejátszás irányát is megváltoztathatja.
			\begin{enumerate}[itemsep=-1ex]
				\item A REV gombal.
				\item A jobbra és balra gombokkal.
			\end{enumerate}
			\item Elindítja a lejátszást.
			\begin{enumerate}[itemsep=-1ex]
				\item A Play gombal.
				\item A szóköz billentyűvel.
			\end{enumerate}
		\end{enumerate}
		\\
		\hline
	\end{tabular}
	\caption{Sztori: A lejátszás elindítása egy tetszőleges sebességgel.}
	\label{tab:story-play}
\end{table}
